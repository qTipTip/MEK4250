\chapter[Convection--Diffusion]{Discretization Of Convection-Diffusion}

\begin{problem_text}
   Derive a proper variational formulation of the convection-diffusion problem.
   Derive sufficient conditions that make the problem well posed.  Discuss why
   oscillations appear for standard Galerkin methods and show how Streamline
   diffusion / Petrov--Galerkin methods resolve these problems. Discuss also
   approximation properties in light of Cea’s lemma.
\end{problem_text}

\section{Finite Element Formulation}

The convection diffusion problem is given as:
\begin{align}
    -\mu\lapl u + \omega \cdot \grad u &= f \text{ in } \Omega, \\
                             u &= g \text{ on } \Gamma_D.
\end{align}
Here \( u \) is the unknown, \( \mu \) is the diffusivity, and \( \omega \) is
a velocity.  We associate to this problem the bilinear operator \( a \colon
\trial \times \test \to \R \) and the linear operator \( L \colon \test \to \R
\) given by
\begin{align}
    a(u, v) &\coloneqq \langle -\mu\lapl u, v \rangle + \langle \omega\cdot
    \grad u, v \rangle;\\
    L(v) &\coloneqq \langle f, v \rangle.
\end{align}

\subsection{Weak Formulation}

We can now consider the \emph{weak formulation} of the Convection-Diffusion
problem. That is: Find \( u \) in \( \trial \) such that:
\begin{equation}
    a(u, v) = L(v) \text{ for all } v \in \test.
\end{equation}
Again, we need to later properly determine the spaces \( \trial \) and \( \test
\), however, as always, we require \( v = 0 \) on \( \Gamma_D \) for all \( v
\in \test \).  In order to get rid of the Laplacian term, we employ the
Gauss--Green lemma, yielding:
\begin{align}
a(u, v) &= \langle \mu \grad u, \grad v \rangle - \underbrace{\int_{\Gamma_D}
\mu \frac{\partial u}{\partial x} v \cdot n \, dS}_{\mathclap{\text{This is zero due to
\( v \in \test\)}}} + \langle \omega \cdot \grad u, v \rangle \\
&= \langle \mu \grad u, \grad v \rangle +  \langle \omega \cdot \grad u, v \rangle.
\end{align}
We again only require one derivative for \( u \) and \( v \) so we let
\begin{align}
    \trial = H^1_g(\Omega) \subseteq H^1(\Omega), \\
    \test = H^1_0(\Omega) \subseteq H^1(\Omega),
\end{align}
as in the Poisson problem. We delay the finite element formulation until after
the Streamline diffusion / Petrov--Galerkin method has been discussed.

We now need to verify that this abstract problem is well posed.

\section{Well Posedness of Weak Formulation}

While our problem is not \emph{homogeneous} in the Dirichlet conditions, we can
reduce it to a homogeneous problem. The reason for doing this is to employ the
Poincar\'e inequality, which only holds for \( H^1_0 (\Omega) \). Again, the
Lax--Milgram theorem gives sufficient conditions for the problem being well
posed. We have two cases for this specific problem:
\begin{inparaenum}[(i)]
    \item Incompressible flow, \( \div \omega = 0\); or
    \item compressible flow, \( \div \omega \neq 0\).
\end{inparaenum}
We deal with these two cases separately.
Furthermore, for simplicity, we define
\begin{align}
    b(u, v) &\coloneqq \langle \mu \grad u, \grad v \rangle,\\
    c_\omega(u, v) &\coloneqq \langle \omega \cdot \grad u, v \rangle,
\end{align}
and note that \(a(u, v) = b(u, v) + c_\omega(u, v)\).

\subsection{Incompressible Flow}
\label{sub:incompressible_flow}

For the incompressible case, we have \( \div \omega = 0 \). In addition to this
assumption, we also assume that the flow velocities are bounded, i.e., \(D_\omega \coloneqq \|
\omega \|_\infty < \infty \). It can then be shown that the bilinear form \(
c_\omega(u, v) \) is \emph{skew-symmetric}, that is
\begin{equation}
    c_\omega(u, v) = - c_\omega(v, u).
\end{equation}
We now show that the conditions in the Lax--Milgram theorem is satisfied.

\paragraph{Coercivity of \( a \):}
Using the skew-symmetric property of \( c_\omega(u, v) \), we get that \(
c_\omega(u, u) = -c_\omega(u, u)\) which implies that \( c_\omega(u, u) = 0 \).
Therefore, we have
    \begin{equation}
        a(u, u) = b(u, u) + c_\omega(u, u) = b(u, u).
    \end{equation}
    So, \(a\) is coercive, as 
    \begin{equation}
        b(u, u) = \mu\int_\Omega (\grad u)^2 \, d\Omega \geq
        \mu\left(\int_\Omega \grad u \, d\Omega\right)^2 = \mu|u|_1^2.
    \end{equation}

\paragraph{Boundedness of \(a\):} 

Applying the Cauchy--Schwartz inequality we have
\begin{align}
    a(u, v) &= \langle \mu\grad u, \grad v \rangle + \langle \omega \cdot \grad u, v \rangle \\
            &\leq |\mu|\|\grad u\|_0 \|\grad v\|_0 + \|\omega \cdot \grad u \|_0 \| v \|_0
    \intertext{Using the assumption of bounded flow velocities; and
    that the problem has been reduced to homogeneous Dirichlet
    conditions --- so we can apply the Poincar\'e inequality with domain dependent factor \( C_\Omega\) --- we get:}
    &\leq |\mu| |u|_1 |v|_1 + D_\omega |u|_1 \| v \|_1 \\
    &\leq(\mu + D_\omega C_\Omega) |u|_1|v|_1.
\end{align}
Consequently, \(a\) is bounded.

\paragraph{Boundedness of \( L \):}

Applying the Cauchy--Schwartz inequality we get
\begin{equation}
    L(u, v) = \langle f, v \rangle \leq \| f \|_0 \| v\|_0 \leq \|f\|_1 \|v\|_1.
\end{equation}

The Lax--Milgram conditions are satisfied, hence the weak formulation of the
convection-diffusion problem is well posed. In addition, the solution \( u \)
satisfies
\begin{equation}
    \| u \|_1 \leq \frac{\mu + D_\omega C_\Omega}{\mu}\|f\|_{-1}.
\end{equation}

\subsection{Compressible Flow}
\label{sub:compressible_flow}

In the case where the flow is compressible, i.e., \( \div \omega \neq 0 \), we
need to put some extra restrictions on the flow velocities \( \omega \) in
order to ensure well posedness. This is because in the general case, we have \(
c_\omega(u, u) \neq 0 \). The coercivity of \(a \) was the only property where
we assumed incompressibility, hence the two other properties remain the same.

\paragraph{Coercivity of \(a\) with compressible fluids:}

If \( D_\omega C_\Omega \leq B\mu \) where \( B < 1 \) we obtain
\begin{align}
    a(u, u) &= \langle \mu \grad u, \grad u \rangle + \langle \omega \cdot \grad u, u \rangle \\
            &\geq \mu (1 - D_\omega)\|u\|_1^2,
\end{align}
however, it is not clear to me exactly how this result is obtained.

\section{Oscillations in the Solution}
\label{sec:oscillations_in_the_solution}

Assume for now we are working with the one dimensional convection diffusion
problem on a mesh with \( h \) denoting the largest mesh element. Solving this
with first order Lagrangian elements corresponds to the central finite
difference scheme
\begin{equation}
    -\frac{\mu}{h^2} [u_{i+1} - 2u_i + u_{i-1}] - \frac{\omega}{2h} [u_{i+1} - u_{i-1}] = 0
\end{equation}
for \( i = 1, \ldots, N-1 \), where \( N \) denotes the number of elements.
Assume the boundary conditions are \( u_0 = 0 \) and \( u_N = 1\). Examining
the above expression we see that in the limit \( \mu \to 0 \) that the scheme
reduces to
\begin{equation}
    \frac{\omega}{2h}[u_{i+1} - u_{i - 1}] = 0
\end{equation}
for \(i = 1, \dots, N\) with \(u_0 = 0\) and \(u_N = 1 \). Here we see that \(
u_{i+1} \) is coupled to \( u_{i-1} \) but not \( u_i \). This means that we
may get a numerical solution consisting of two sequences \((u_{2i})_i\) and
\((u_{2i+1})_i\) that have no relation to each other. This may very well cause
oscillations in the solution.

\subsection{Finite Difference Upwinding}

One remedy is to introduce the concept of \emph{upwinding}. Instead of using a
central finite difference scheme as above, one employs either a forward or a backward
first order scheme, based on the velocity \( \omega\). That is:
\begin{align}
    u'(x_i) \approx \frac{1}{h}[u_{i+1} - u_i] \text{ if } \omega < 0, \\
    u'(x_i) \approx \frac{1}{h}[u_{i} - u_{i-1}] \text{ if } \omega > 0.
\end{align}
This upwinding scheme can be seen as a special case of \emph{artificial
diffusion}, where one solves the ``artificial'' problem
\begin{equation}
    -(\mu + \varepsilon)\lapl u + \omega \cdot \grad u = f,
\end{equation}
with \( \varepsilon > 0 \) some arbitrary real number. In particular, choosing
\( \varepsilon = h / 2 \) one regains the upwinding scheme mentioned above.

The fact that the finite element method coincides with the finite difference
method in the case of one dimensional convection diffusion, and first order
Lagrangian elements is not something that holds in general. Our question in the
following is then: How do we implement artificial diffusion in our finite
element method? This leads us to the Streamline diffusion / Petrov--Galerkin
methods.

\section{Streamline Diffusion / Petrov--Galerkin}
\label{sec:streamline_diffusion_petrov_galerkin}

Our goal here is to add artificial in a consistent way that does not changes
the solution as \( h \) tends to zero. It turns out that naively adding
artificial diffusion to our current finite element formulation does not give us
what we want. We first examine why. 

\subsection{Naive Artificial Diffusion}
\label{sub:naive_artificial_diffusion}

Recall that our problem reads: Find \( u \in \trial\) such that
\begin{align}
    \langle \mu \grad u, \grad v \rangle + \langle \omega \cdot \grad u, v
    \rangle = \langle f, v \rangle \text{ for all } v \in \test.
\end{align}
Replacing \( \mu \) by \( \mu + \varepsilon \) yields a new bilinear operator \( \tilde{a} \):
\begin{equation}
    \tilde{a}(u, v) \coloneqq \langle \mu \grad u, \grad v \rangle + \langle
    \varepsilon \grad u, \grad v \rangle + \langle \omega \cdot \grad u, v
\rangle.
\end{equation}
This can be written succinctly as 
\begin{equation}
    \tilde{a}(u, v) = a(u, v) + \varepsilon \langle \grad u, \grad v\rangle.
\end{equation}
If we let \( \varepsilon = h / 2 \) we see that in the limit as \( h \to 0 \),
we have \( \tilde{a}(u, v) \to a(u, v) \), and the scheme is consistent in this
sense. However, it is not \emph{strongly consistent} as it does not satisfy
the Galerkin-orthogonality as \( a \) does: 
\begin{equation}
    a(u - u_h, v) = 0 \text{ for all } v \in \test_h,
\end{equation}
namely that this equation is zero for \emph{all} discretization, and not just
in the limit.

We can however make the scheme strongly consistent by employing different
spaces for the test functions and the trial functions.

\subsection{Petrov--Galerkin method}

The only difference between the Petrov--Galerkin and the standard Galerkin
formulation is that the trial and test functions differ. In the standard
Galerkin method, the same basis is used for both test and trial functions.  In
the Petrov-Galerkin method the test functions are tailored to ensure a
strongly consistent scheme.

\subsection{Finite Element Formulation}
\label{sub:finite_element_formulation_diffusion}

Letting \( \test_h\) and \(\trial_h\) denote the finite dimensional subspaces
of \( \test \) and \( \trial \) respectively. Assume these are given by bases
\( (\psi_j)_{i=1}^M \) and \( (\varphi_i)_{i=1}^N \).  The finite element
formulation then gives rise to a linear system
\begin{equation}
    \A \c = \b,
\end{equation}
where the matrix elements are given as
\begin{align}
    A_{i, j} &= \langle \mu \grad \varphi_i , \grad \psi_j \rangle + \langle \omega \cdot \grad \varphi_i, \psi_j \rangle, \\
    b_j &= \langle f, \psi_j \rangle.
\end{align}
How do we choose the basis \( (\psi_j)_{j=1}^M \) such that we can add
diffusion consistently? It turns out that setting
\begin{equation}
    \psi_j \coloneqq \varphi_j + \varepsilon\omega \cdot \grad \varphi_j
\end{equation}
does the trick. Expanding the matrix elements with this new basis yields
\begin{align}
    A_{i, j} &= \langle \mu \grad \varphi_i , \grad\varphi_j \rangle
    + \varepsilon\langle \mu \grad \varphi_i, \grad (\omega\cdot\grad\varphi_j)\rangle
    + \langle \omega \cdot \grad \varphi_i, \varphi_j \rangle
    + \varepsilon\langle \omega \cdot \grad \varphi_i,  \omega \cdot \grad \varphi_j\rangle\\
    b_j &= \langle f, \varphi_j \rangle + \varepsilon\langle f,  \omega \cdot \grad \varphi_j \rangle.
\end{align}
The terms not containing \( \varepsilon \) correspond to the standard Galerkin
method. 

\section{Error estimates}
\label{sub:chp_2_error_estimates}

We consider two different error estimates. One being an estimate for the error
in the standard Galerkin approximation. The other one is tailored for the
Streamline diffusion / Petrov--Galerkin method.

\subsection{Standard Galerkin Method}

We employ the same trick as in \vref{ssec:error_estimate_cea} where we assume
boundedness of \(a\). Let \( u_h \) be the computed solution. Using the
coercivity of the bilinear form \( a \) and the Galerkin orthogonality property
we get
\begin{align}
    \|e\|_1^2 &\leq \frac{1}{\alpha} a(e, e) = \frac{1}{\alpha} a(e, u - v + v - u_h)\\
              &= \frac{1}{\alpha}a(e, u - v) \leq \frac{C}{\alpha}\| e\|_1 \| u - v \|_1
\end{align}
for all \( v \in \test \).
Dividing both sides yield
\begin{equation}
    \| e \|_1 \leq \frac{C}{\alpha}\| u - v \|_1.
\end{equation}
The Bramble--Hilbert lemma yields a bound on the interpolation error of a
certain type of interpolation operator, denoted \( \pi_{p, h} u \) of order \(
p \). This can be combined with the above error estimate to yield
\begin{equation}
    \| e \|_1 \leq \frac{C}{\alpha}\| u - \pi_{p, h} u \|_1 \leq \frac{CB}{\alpha} \| h^p u \|_{p+1}.
\end{equation}
Recall that the constant \( \alpha \), coming from the coercivity of \(a \), is
given as \( \alpha = \mu ( 1 - D_\omega) \). If \( \mu \) is very small, i.e.,
in convection dominated problems,  then our error bound becomes very bad. This
can be fixed by looking at a more specifically tailored error estimate.

\subsection{Petrov--Galerkin method}

Introduce the \emph{SUPG-norm} defined as follows:
\begin{equation}
    \| u \|_{\mathrm{SUPG}} \coloneqq \left(h \| \omega \cdot \grad u \|^2 + \mu | \grad u |^2 \right)^{1 / 2}
\end{equation}
It turns out that solving the Petrov--Galerkin problem on a finite element
space of order \( 1 \) with the same assumptions as above, then
\begin{equation}
    \| u - u_h\|_{\mathrm{SUPG}} \leq Ch^{3/2} \|u\|_2.
\end{equation}
This is stated without proof.
