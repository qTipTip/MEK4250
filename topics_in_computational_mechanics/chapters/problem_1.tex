
\chapter[Poisson Equation]{Weak Formulation And Finite Element Error Estimation}

\begin{problem_text}
    Formulate a finite element method for the Poisson problem with a variable
    coefficient \( \kappa \colon \Omega \to \R^{d\times d}\). Show that the
    Lax--Milgram theorem is satisfied. Consider extensions to e.g. the
    convection-diffusion and the elasticity equation. Derive \emph{a priori}
    error estimates for the finite element method in the energy norm. Describe
    how to perform an estimation of convergence rates.
\end{problem_text}

\section{Finite Element Formulation}

The Poisson problem is formulated as:
\begin{align}
    -\div (\kappa \grad u) &= f \text { in } \Omega,\\
    u &= u_0 \text{ on } \Gamma_D,\\
    -\kappa \grad u \cdot n &= g \text{ on } \Gamma_N.
\end{align}
Here \( u \) denotes the unknown field. We associate to this strong formulation
of the problem, the bilinear operator \( a\colon \test \times \trial \to \R\),
and the linear operator \( L \colon \test \to \R\) as follows:
\begin{align}
    a(u, v) &\coloneqq \langle -\div(\kappa \grad u), v \rangle, \\
    L(v) &\coloneqq \langle f, v \rangle.
\end{align}

\subsection{Weak Formulation}
\label{sub:weak_formulation}

We can therefore consider instead the \emph{weak formulation} of the Poisson
problem, which is: Find \( u \in \trial \) such that 
\begin{equation}
    a(u, v) = L(v) \text{ for all } v \in \test.
\end{equation}
We need to place some requirements to the spaces \(\trial \) and \(\test\) in
order to satisfy the boundary conditions. However, one requirement we impose
immediately is that \(v = 0\) on \( \Gamma_D \) for all \( v \in \test\).  What
these spaces should be is not immediate from the current formulation. Throughout
the following, \( \langle \cdot, \cdot \rangle \) denotes the \( L^2 \) inner
product on \( V \). Expanding \( a \) using among others the Gauss--Green
lemma yields:
\begin{align}
    a(u, v) &= \langle -\div(\kappa \grad u), v \rangle = -\langle \kappa \lapl u, v \rangle\\
            &= \langle \kappa \grad u, \grad v \rangle - \int_{\partial\Omega} - v \kappa  \frac{\partial u}{\partial x}  \cdot n \, dS.
\end{align}
The boundary integral above can be rewritten using the partitioning of the boundary:
\begin{equation}
    \int_{\partial\Omega} -v \kappa  \frac{\partial u}{\partial x}  \cdot n \, dS =
    -\int_{\Gamma_D} v \kappa \frac{\partial u}{\partial x} \cdot n \, dS + \int_{\Gamma_N} -v \kappa \frac{\partial u}{\partial x} \cdot n \, dS.
\end{equation}
Applying the boundary conditions over respective boundaries and condition on \(
\test \) we get in total that:
\begin{equation}
    a(u, v) = \langle \kappa\grad u, \grad v \rangle - \int_{\Gamma_N} g v \, dS.
\end{equation}
Since the boundary term in \( a \) is independent of \( u \) we decide to
transfer this term to \( L \):
\begin{equation}
    a(u, v) = \langle \kappa\grad u, \grad v \rangle, \quad L(v) = \langle f, v
    \rangle + \int_{\Gamma_N} g v \, dS.
\end{equation}


We now have what we need to determine the spaces \( \test \) and \( \trial \).
Note that we in the weak form only require \( u \) and \( v \) to be once
differentiable. Furthermore, we require \( u \) to reduce to \( u_0 \) on \(
\Gamma_D \) by the boundary conditions, while we need \( v \) to vanish
identically at \(\Gamma_D\). We therefore decide on the test and trial spaces
\begin{align}
    \trial &= H^1_g(\Omega) \coloneqq \left\{ u \in H^1(\Omega) : u = g \text{ on } \Gamma_D \right\} \subseteq H^1(\Omega),\\
    \test &= H^1_0(\Omega) \coloneqq \left\{ u \in H^1(\Omega) : u = 0 \text{ on } \Gamma_D \right\} \subseteq H^1 (\Omega).
\end{align}

\subsection{Finite Element Formulation}
\label{sub:finite_element_formulation}

In order to compute with these spaces, we need to introduce a basis. However,
the spaces might be infinite dimensional, so we approximate by finite subspaces
\( \test_h \) and \(\trial_h\) respectively%
%
\footnote{Much of the finite element theory amounts to determining what the
error in this specific approximation is.}. %
%
Since we have \( \test \subseteq \trial \) we have \( \test_h \subseteq
\trial_h \), and we therefore use the same basis vectors for both test and
trial functions. We seek a solution \( u = \sum_i c_i \varphi_i \) such that
\begin{align}
    \sum_{i=1}^{N} c_i a(\varphi_i, \varphi_j) = L(\varphi_j) \text{ for } j = 1, \dots, M
\end{align}
where \(N, M\) denotes the dimensions of \(\trial_h, \test_h\) respectively.
This determines a linear system
\begin{equation}
    \A \c = \b
\end{equation}
where the matrix entries are determined as follows:
\begin{equation}
    A_{i, j} = a(\varphi_i, \varphi_j), \quad b_j = L(\varphi_j).
\end{equation}

\section{Well Posedness of Weak Formulation}
\label{sec:well_posedness_of_weak_formulation}

In this section we discuss whether the problem in its weak formulation is
in fact well posed. Does there exist a solution, and if it does, is this
solution unique? The Lax--Milgram theorem provides sufficient conditions for
this problem to be well posed, and hence the solution to exist and be unique.
We verify the three properties in turn:
\begin{description}
    \item[Boundedness of \( a \):]
        Let \( u, v \in H^1(\Omega) \). Then we have
        \begin{equation}
            a(u, v) = \langle \kappa \grad u, \grad v\rangle
            \underbrace{\leq}_{\mathclap{\text{Using the Cauchy--Schwartz inequality}}} 
            \overbrace{\|\kappa\|}^{\mathclap{\text{subordinate matrix norm}}} 
            \| \grad u \|_0 \| \grad v \|_0
            = \|\kappa\| \| u \|_1 \| v \|_1
        \end{equation}
        This is close to proving boundedness of \( a \), however, I do not know
        how to deal with the \( \| \kappa \| \). However, for \( \kappa = 1\)
        this proves the claim.

    \item[Coercivity of \(a \):]
        Let \( u \in H^1 (\Omega) \) and consider the following:
        \begin{equation}
            \| u \|_1^2 \overset{\text{def}}{=} \| u \|_0^2 + \| \grad u \|_0^2
            \underbrace{\leq}_{\mathclap{\text{Using the Poincar\'e inequality
            on } H^1_0(\Omega)}} \geq (C^2 + 1) \| \grad u \|_0^2
        \end{equation}
        This shows that \( a(u, u) \ (C^2 + 1)^{-1} \| u \|_1^2 \) in the
        case where \( \kappa = 1 \) identically. 

    \item[Boundedness of \( L \):]
        \begin{align}
            L(v) &= \langle f, v \rangle + \int_{\Gamma_N} g v\, dS \\
            \shortintertext{Using Cauchy--Schwartz on both terms yields }
                 &\leq \| f\|_0 \| v \|_0 + \| g \|_{L^2(\Gamma_N)} \| v \|_{L^2(\Gamma_N)} ;\\
            \shortintertext{Using the fact that \( v \) is defined on the
            entirety of \( \Omega \), the \( L^2 \) norm is certainly larger
            over the whole domain}
                 &\leq \| f \|_0 \| v \|_0 + \|g\|_{L^2(\Gamma_N)} \| v \|_0;\\
            \shortintertext{Using that the \( H^1 \) norm greater than or equal
            to the \( L^2 \) norm}
                 &\leq (\| f\|_0 + \| g \|_{L^2(\Gamma_N)}) \| v \|_1,\\
                 &= D \| v \|_1.
        \end{align}
        This proves the boundedness of \( L \).
\end{description}
This means that the weak formulation of the Poisson problem satisfies the
Lax--Milgram theorem, hence is well posed. In addition, the solution \( u \) satisfies
\begin{equation}
    \| u \|_1 \leq \frac{(C^2 + 1)^{-1}}{\|\kappa\|}\| f \|_{-1}.
\end{equation}

\section{Extensions to Other Problems}
\label{sec:extensions_to_other_problems}

Recall that the equation of linear elasticity is given as
\begin{equation}
    \label{eq:lin_elast_poisson}
    - 2 \mu ( \div \varepsilon(u)) - \lambda \grad(\div u) = 0
\end{equation}
where \(\varepsilon(u) \coloneqq (\grad u + (\grad u)^T) / 2 \) is the strain
tensor. Ignoring the second term, first term is supposedly a
Poisson-equation\footnote{This needs to be verified.}.  However, in order to
check whether we can apply the results found for the abstract Poisson-problem
we need to determine what the \( \grad (\div u) \) in the second term is. The
Helmholtz decomposition theorem states that any \( u \) in \( L^2(\Omega) \)
can be decomposed into a curl-free part \( \psi \)  and a divergence-free part
\( \varphi \):
\begin{equation}
    u = \psi + \varphi 
\end{equation}
where \(\div \varphi = 0\) and \(  \curl \psi = 0 \).  In the light of this, we
can consider the two special cases where either \( \varphi \) or \( \psi \) are
zero. 
\begin{enumerate}[(i)]
    \item Assume that \( \varphi = 0 \). This means that \( u = \psi \)
        consists only of the divergence free part. We then have that
        \(\grad(\div u) = 0 \). Consequently, the second term vanishes in
        \cref{eq:lin_elast_poisson}.
    \item Assume that \( \psi = 0 \), i.e., \( u \) consists only of the curl
        free part. Then using the identity
        \begin{equation}
            \lapl u = \grad(\div u) - \curl(\curl v)
        \end{equation}
        we see that \( u = \grad(\div(u)) \). Hence,
        \cref{eq:lin_elast_poisson} reduces to a Poisson problem.
\end{enumerate}
In these two special cases, we have that the linear elasticity problem reduces
to a Poisson problem of the form
\begin{equation}
    -(\mu + \lambda)\lapl w = f.
\end{equation}
for some source term \( f \).

\section{A Priori Error Estimates}
\label{sec:a_priori_estimates}

In the following, we work over arbitrary test and trial spaces \( V \), as
these estimates are general.

\subsection{Energy Norm}
We now discuss some a priori estimates. Recall that any arbitrary inner product
induces a norm by
\begin{equation}
    \| x \| \coloneqq \sqrt{ \langle x, x \rangle }.
\end{equation}
It turns out that the bilinear form \( a \) may, or may not, constitute an
inner product. If it does, then we may talk about the norm induced by this
bilinear form.  We call this the \emph{energy norm}. We first need to verify
that \( a \) does indeed constitute an inner product. The only condition that
is not trivial is the coercivity, however we already proved this for the
Lax--Milgram theorem. We therefore define the energy norm
\begin{equation}
    \|u\|_E \coloneqq \sqrt{a(u, u)}.
\end{equation}

Consider now the error \( e \coloneqq u - u_h \). Let \(v \in \test\) be
arbitrary. In the energy norm, we have
\begin{align}
    \| e \|_E^2 = a(e, e) &= a(e, u - u_h) = a(e, u - v + v - u_h) \\
                          &= a(e, u - v) + a(e, \underbrace{u_h - v}_{\in \test}) = a(e, u-v).
\end{align}
Using the Cauchy--Schwartz inequality, we have that
\begin{equation}
     \| e \|_E^2 \leq \|e\|_E \|u - v\|_E \implies \| e \|_E \leq \|e - v\|_E
\end{equation}
for all \( v \in \test \). This does however not give any sharp bounds, so it
is hard to quantify exactly what magnitude the error has. We can however
combine this result with an interpolation error estimate:
\begin{equation}
    \| e \|_E \leq \| u - \pi_{q, h} u\|_E \leq C(q) \| h^{q + 1}
    D^{q+1} u\|.
\end{equation}
Here, \( \pi_{q, h} u\) denotes the \( q \)-th order interpolant to \( u \),
and \( h \) is the maximum mesh size.


\subsection{Error estimate without coercivity of \(a\)}
\label{ssec:error_estimate_cea}
The symmetry of \( a \) is quite a strong requirement, and this is not always
satisfied. In this section, we do \emph{not} assume that \( a \) is symmetric.
In the cases where \( a \) is bounded however we can consider the error in the
vector space norm as follows. Let \( v \in \test \) be arbitrary. Then using
the coercivity of \( a \) we have:

\begin{align}
    \| e \|_V^2 &\leq \frac{1}{\alpha} a(e, e) = \frac{1}{\alpha}a(e, u- v + v - u_h) \\
                &= \frac{1}{\alpha} a(e, u - v)
                \underbrace{\leq}_{\mathclap{\text{Using boundedness of \(a
        \)}}} \frac{D}{\alpha}\| e \|_V \| u - v \|_V.
\end{align}
Dividing by \( \|e\|_V \) and combining with an interpolation error estimate,
as we did above, 
\begin{equation}
    \|e \|_V \leq \frac{D C(q)}{\alpha} \|h^{q+1} D^{q+1}u \|.
\end{equation}

\section{Error Approximation}

In the following we assume we solve a constructed problem with known solution.
Let \( N \) denote the number of basis elements over the domain. Then we can
consider the error as a function of \( N \). Using degree \( p \) elements over
a mesh with maximal mesh size of \( h \). Let \( u \) denote the analytic
solution, and \( u_N \) the computed solution with \( N \) basis elements.
Denote by \( e_N \) the error in the corresponding approximation.
We wish to approximate the convergence rate \( \beta \) in the following:
\begin{equation}
    \| e \|_V \leq \| h^\beta D^\beta u \|.
\end{equation}
We can rewrite this as a linear equation in \( h \) with slope \(\beta\) and
constant term \( \log(D^\beta u) \). Computing \( e_N \) for various \( N \) we
can find a linear regression line through the data and approximate \( \beta \).
