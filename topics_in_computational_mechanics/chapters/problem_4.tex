\chapter[Navier--Stokes]{Discretization Of Navier--Stokes}

\begin{problem_text}
    Explain the difference between operator splitting and algebraic splitting
    in the context of incompressible Navier--Stokes equations. Show
    disadvantages of operator splitting schemes associated with boundary
    conditions. Explain the advantage with operator splitting schemes as
    compared to algebraic splitting in the context of Richardson iteration and
    spectral equivalence.
\end{problem_text}

Recall that the incompressible Navier--Stokes equations are given by
\begin{equation}
    \label{eq:NS}
    \frac{\partial v}{\partial t} + v \cdot \grad v = -\frac{1}{\varrho} \grad p + \nu \grad^2 v + g,
\end{equation}
with the equation of continuity, or \cemph{incompressibility constraint},
\begin{equation}
    \label{eq:NSc}
    \div v = 0.
\end{equation}
Here \( v \) is the velocity field, \( p \) is the pressure, \( \varrho \) is
the fluid density, and \( g \) is an umbrella term describing body forces.

\section{Operator Splitting}
\label{sub:operator_splitting}

The basic operator splitting algorithm can be summarized in three steps.  These
steps will be motivated in the following.
\begin{enumerate}[(i)]
    \item Compute the velocity prediction \( v^\star \) from an explicit equation involving the previous velocity \( v^n \).
    \item Solve the obtained Poisson equation for the pressure difference in time.
    \item Compute the new velocity \( v^{n+1} \) and the new pressure \(
        p^{n+1} \) from explicit equations.
\end{enumerate}

\subsection{Explicit Scheme}
Starting with a forward step in time in \cref{eq:NS}:
\begin{equation}
    \label{eq:forward_step}
    v^{n+1} = v^n - \dd t \left(v^{n} + \frac{1}{\varrho} \grad p^n - \nu \grad^2 v^n - g^n\right).
\end{equation}
This cannot be the velocity at time level \( n + 1 \) because it does not
satisfy the equation of continuity \cref{eq:NSc}, i.e., \(\div v^{n+1} \neq 0
\). We instead use this as an intermediate step, denoted \( v^\star \), in
computing the new velocity. We may attempt to use the incompressibility
constraint to compute a correction term \( v^\mathrm{c} \) such that \( v^{n+1}
= v^\star + v^\mathrm{c} \). In order to gain some control over the pressure
data in the equation, we also introduce the factor \( \beta \) to the pressure
term:
\begin{equation}
    v^\star = v^n - \dd t \left(v^n + \frac{\beta}{\varrho}\grad p^n - \nu
    \grad^2 v^n - g^n \right).
\end{equation}
The computed velocity \( v^{n+1} \) should also solve \cref{eq:forward_step}
where the pressure is evaluated at time level \(n + 1\). That is 
\begin{equation}
    v^{n+1} = v^n - \dd t \left(v^{n} + \frac{1}{\varrho} \grad p^{n+1} - \nu \grad^2 v^n - g^n\right).
\end{equation}
Subtracting these two equations yield an expression for \( v^\mathrm{c} \):
\begin{align}
    v^\mathrm{c} &\coloneqq v^{n+1} - v^\star = -\frac{\dd t}{\varrho}
    \grad\left(  p^{n+1} - \beta p^n \right), 
    \shortintertext{or equivalently:}
    v^{n+1} &= v^\star - v^\mathrm{c} = v^\star - \frac{\dd t}{\varrho} \grad \left( p^{n+1} - \beta p^n \right).
\end{align}
Setting \( \varphi \coloneqq p^{n+1} - \beta p^n\) and requiring that \( \div
v^{n+1} = 0 \) yields a Poisson equation in \( \varphi \):
\begin{equation}
    \grad^2 \varphi = \frac{\varrho}{\dd t} \div v^{\star}.
\end{equation}
After solving this for \( \varphi \) we may update our solution in both
velocity and pressure as follows:
\begin{align}
    \label{eq:update}
    p^{n+1} &= \beta p^n + \varphi, \\
    v^{n+1} &= v^\star - \frac{\dd t}{\varrho} \grad \varphi.
\end{align}

\subsubsection{Boundary Conditions}
\label{par:boundary_conditions}

A question arises, namely: How do we solve the Poisson equation for the
pressure difference \( \varphi \)? We are short on boundary conditions. Two
remedies are proposed:
\begin{enumerate}[(i)]
    \item Computing \( \partial p / \partial n\) from \cref{eq:NS} by
        multiplying the equation by the unit normal vector. This gives us
        Neumann boundary conditions for the pressure difference, \( \partial
        \varphi / \partial n \).
    \item Since \( v^{n+1} \) is supposed to satisfy the Dirichlet boundary
        conditions, then from \cref{eq:update} we must have
        \begin{equation}
            \grad \varphi |_{\partial \Omega} = \frac{\dd t}{\varrho} (v^{n+1} - v^\star)|_{\partial\Omega} = 0.
        \end{equation}
        Hence \( \varphi \) must be constant on the boundary.
\end{enumerate}

\subsection{Implicit Scheme}
\label{par:implicit_scheme}

In the above derivations an explicit scheme was used for the stepping. It is
perfectly reasonable to instead use implicit schemes, like a \( \theta
\)-scheme. This will generally lead to a solution that involves an
advection-diffusion equation, a Poisson equation, and then performing two
implicit updates of the velocity and pressure. We will not go into detail here.

\section{Algebraic Splitting}
\label{sec:algebraic_splitting}
 
Note that in the operator splitting scheme, we discretize in time before
discretizing in space. This leads to the need for more boundary conditions in
order to solve the Poisson-problem for the pressure difference. An alternative
approach discretizes in space before we discretize in time. This removes the
need for construction of additional boundary conditions as these are baked into
the algebraic constraints. Discretizing the spatial operators in
\cref{eq:NS,eq:NSc} using for instance a finite element method yields a set of
linear systems:
\begin{align}
    \bsy{M}\dot{\bsy{u}} + \bsy{K}(\bsy{u}) \bsy{u} &= - \bsy{Q}\bsy{p} + \A\bsy{u} + \bsy{f}, \\
    \bsy{Q}^T\bsy{u} &= 0.
\end{align}

\subsection{Explicit Schemes}

We can employ the same methodology as in \cref{sub:operator_splitting}. Compute
a tentative velocity \( \u^\star \) from 
\begin{equation}
    \bsy{M}\u^\star = \bsy{M}\u^n + \dd t \left( -\bsy{K}(\u^n)\u^n - \beta
    \bsy{Q}\bsy{p}^n+ \A \u^n + \bsy{f}^n\right).
\end{equation}
Again, this tentative velocity does not necessarily satisfy the equation of
continuity, \(\bsy{Q}^T\u = 0\). We use \( \u^\star \) to compute a correction
term \( \u^\mathrm{c} \) such that \( \u^{n+1} \coloneqq \u^\star +
\u^\mathrm{c} \) satisfies \( \bsy{Q}^T \u^{n+1} = 0 \). In analogous fashion
to the operator splitting method we can formulate a discrete Poisson equation
for the pressure difference \( \varphi \):
\begin{equation}
    \bsy{Q}^T\bsy{M}^{-1}\bsy{Q}\varphi = \frac{1}{\dd t} \bsy{Q}^T\u^\star.
\end{equation}
Solving this for \( \varphi \) we update the pressure and velocity for the next
time level:
\begin{align}
    \bsy{p}^{n+1} &= \beta \bsy{p}^n  + \varphi, \\
    \u^{n+1} &= \u^\star - \dd t \bsy{M}^{-1}\bsy{Q}\varphi.
\end{align}

\subsection{Classical Schemes}
\label{sub:classical_schemes}

We may employ a more general implicit scheme, known as the \(\theta\)-scheme.
Here \( \theta \) is a \emph{blending parameter}, where \( \theta = 0 \)
corresponds to the Backward Euler scheme, \( \theta = 1/2 \) corresponds to the
Crank--Nicolson scheme, and \( \theta = 0\) yields the explicit Forward Euler
Scheme as above. It turns out that these classical schemes can be viewed in the
light of iterative methods. 

Discretizing in time yields what we will refer to as the \emph{fully implicit
system}:
\begin{align}
    \bsy{N}\u^{n+1} + \dd t \bsy{Q}\bsy{p}^{n+1} &= \bsy{q}^{n+1}, \\
    \bsy{Q}^T\bsy{u}^{n+1} &= 0.
\end{align}
Here we have abbreviated as follows:
\begin{align}
    \bsy{N} &\coloneqq \bsy{M} + \theta \dd t \bsy{R}(\u^{n}), \\
    \bsy{R}(\u^n) &\coloneqq \bsy{K}(\u^n) - \A, \\
    \bsy{q} &\coloneqq (\bsy{M} - (1- \theta)\dd t \bsy{R}(\u^n))\u^n + \dd t \bsy{f}^{n}.
\end{align}
Assuming invertibility of \( \bsy{N} \) we can solve this for \( \u^{n+1} \)
--- and inserting this into the second equation yields what is known as the
\emph{Schur complement pressure equation}:
\begin{align}
    \label{eq:schur}
    \bsy{Q}^T\bsy{N}^{-1}\bsy{Q}\bsy{p}^{n+1} = \frac{1}{\dd t}\bsy{Q}^T\bsy{N}^{-1}\bsy{q}.
\end{align}
The system \( \bsy{N}^{-1} \) can often be computationally costly to solve.
Since \( \bsy{N} \) is a sparse matrix, \(\bsy{N}^{-1}\) is dense, so we we
would like some other way of dealing with this equation.  Luckily, we can note
that \cref{eq:schur} is a linear system of the form
\begin{equation}
    \B \bsy{p}^{n+1} = \b.
\end{equation}
A preconditioned Richardson iteration can be formulated for \( \bsy{p}^{n+1} \)
\begin{equation}
    \bsy{p}^{n+1, k+1} = \bsy{p}^{n+1, k} - \bsy{C}_1^{-1}(\B\bsy{p}^{n+1, k} - \b),
\end{equation}
where \(\bsy{C}_1^{-1}\) is a preconditioner similar to \(\bsy{N}^{-1}\)
however in some sense, simpler to solve, and \( k \) is an iteration counter.
For each time level we start the iteration with \( \bsy{p}^{n+1, 0} = \bsy{p}^n
\). For the Schur complement pressure equation this yields an iteration of the
form
\begin{equation}
    \bsy{p}^{n+1, k+1} = \bsy{p}^{n+1, k} - \bsy{C}_1^{-1}(\bsy{Q}^T\bsy{N}^{-1}\bsy{Q}\bsy{p}^{n+1, k} - \frac{1}{\dd t}\bsy{Q}^T \bsy{N}^{-1} \bsy{q}).
\end{equation}
Formulating a similar procedure for the velocity, with preconditioner
\(\bsy{C}_2^{-1}\) we end up with a preconditioned system similar to the fully
implicit system:
\begin{equation}
    \begin{bmatrix}
        \bsy{C}_1 & 0 \\
        0 & \bsy{C}_2 
    \end{bmatrix}
    \begin{bmatrix}
        \bsy{N} & \bsy{Q} \\
        \bsy{Q}^T & 0
    \end{bmatrix}
    \begin{bmatrix}
        \bsy{u}  \\
        \bsy{p}
    \end{bmatrix}
    = 
    \begin{bmatrix}
        \bsy{q} \\
        \bsy{0}
    \end{bmatrix}.
\end{equation}
