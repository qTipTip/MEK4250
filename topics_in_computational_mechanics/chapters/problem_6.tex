\chapter{Linear Elasticity}

\begin{problem_text}
    Describe a variational formulation of the linear elasticity equation.
    Discuss appropriate boundary conditions, problems associated with rigid
    motions, and Korn's lemma. Discuss the locking phenomenon and mixed
    formulations that avoid locking.
\end{problem_text}

The topic of linear elasticity deals with deformation of an elastic body.
By Newton's second law we have that
\begin{align}
    \div \sigma &= f, \text{ in } \Omega,\\
    \sigma \cdot n &= g, \text{ on } \Gamma_N
\end{align}
where \( f \) and \( g  \) denote body and surface forces respectively.  For
small deformations of an isotropic media, we may approximate this using Hooke's
law, which states that
\begin{equation}
\sigma = 2 \mu \varepsilon(u) + \lambda \tr{\varepsilon(u)}\delta.
\end{equation}
Here, \( \varepsilon (u) \) denotes the symmetric gradient
\begin{equation}
    \varepsilon(u) \coloneqq \frac{\grad u + (\grad u)^T}{2}.
\end{equation}
Combining Newton's second law, and Hooke's law, yields the linear elasticity equation:
\begin{equation}
    \label{eq:linelast}
    -2\mu(\div \varepsilon(u)) - \lambda \grad(\div u) = f.
\end{equation}
We already saw in \vref{sec:extensions_to_other_problems} that, in some
circumstances, this equation reduces to a Poisson equation. However, in this
section, we discuss it in its generality.

\section{Variational Formulation}
\label{sec:variational_formulation}

We here derive the variational formulation of the linear elasticity equation.
For a standard Galerkin method, we integrate by parts as follows:
\begin{align}
    \langle -2\mu(\div \varepsilon(u), v \rangle - \lambda \langle \grad (\div
        u), v \rangle = \langle f, v \rangle,
    \shortintertext{where Gauss--Green's lemma gives}
     \langle 2\mu \varepsilon(u), \grad v \rangle - \int_{\Gamma_N}2\mu
     \varepsilon(u) \cdot n v \, \dd S \\ + \langle \lambda (\div u), \div v
     \rangle - \langle \lambda(\div u) \cdot n, v \rangle_{\Gamma_N} =
     \langle f, v \rangle.
\end{align}
Simplifying this yields
\begin{equation}
    \langle 2\mu\varepsilon(u), \grad v \rangle + \langle \lambda(\div u), \div v \rangle = \langle f, v \rangle + \langle \sigma \cdot n, v \rangle_{\Gamma_N}.
\end{equation}
Introducing the two operators \( a \) and \( L \):
\begin{align}
    a(u, v) &\coloneqq 2\mu\langle \varepsilon(u), \varepsilon(v)\rangle +
    \lambda \langle \div u, \div v \rangle \\
    L(v) &\coloneqq \langle f, v \rangle + \langle g, v \rangle_{L^2({\Gamma_N})}
\end{align}

\subsection{Boundary conditions}
\label{sub:boundary_conditions}

The natural boundary conditions for the linear elasticity is the pure Neumann
conditions
\begin{equation}
    \sigma \cdot n = g.
\end{equation}
There are a number of subtleties related to the choice of boundary conditions
for the linear elasticity question.  It turns out that having pure Dirichlet
boundary conditions yields a unique solution, and this is formulated in Korn's
lemma. Having mixed Neumann and Dirichlet conditions also works. Having pure
Neumann conditions on the other hand leads to a singular system, and this is
due to the operator \( \div \varepsilon(u) \). This can be seen as analogous to
the case where we have pure Neumann boundary conditions in the Poisson problem.

\section{Rigid Motions}
\label{sec:rigid_motions}

This section deals with the problem of having pure Neumann conditions. The
kernel of \( \div \varepsilon \) is the set of rigid motions. This means that
if an elastic body \( u \) solves the equation of linear elasticity, then any
translations and rotations of \( u \), yielding a new elastic body \( \tilde u
\), then \( \tilde u \) also solves the linear elasticity equation.

As an example in \( \R^2 \), let \( u \) be the rigid motion
\begin{equation}
    \(u(x, y) \) \coloneqq \begin{bmatrix}
        a_1 \\
        a_2
    \end{bmatrix} + b \begin{bmatrix}
        y \\
        -x
    \end{bmatrix}.
\end{equation}
Computing \( \div u \) we see that \( \div u = 0\). Furthermore, 
\begin{equation}
    \div \varepsilon(u) = \frac{1}{2} \left( \begin{bmatrix}
            0 & -1 \\
            1 & 0
    \end{bmatrix} + \begin{bmatrix}
        0 & 1 \\
        -1 & 0
    \end{bmatrix}\right) = 0, 
\end{equation}
hence the linear elasticity equation reduces to \( 0 = f \), meaning that \( f
\) has to be zero. While this is indeed a \emph{solution}, it is not
\emph{unique}, as we never talked about the values \( a_1, a_2 \) and \( b \).
In order to describe a unique solution, we need to prescribe Dirichlet
conditions to fixate these numbers. This means that in the case where we only
have pure Neumann boundary conditions, we need to remove the rigid motions from
the solution space.

\section{Korn's Lemma}
\label{sec:korn_s_lemma}

The following lemmas provide suitable conditions for solvability of the linear
elasticity equation. Denote \( \mathcal{R} \coloneqq \left\{ u \in H^1(\Omega)
: u \text{ is a rigid motion} \right\}\). Starting with pure Neumann conditions
we have the following: For all \( u \in H^1(\Omega) \setminus \mathcal{R} \) we
have that
\begin{equation}
    \label{eq:korn_neumann}
    \| \varepsilon(u) \| \geq C \| u \|_1.
\end{equation}
For the more special case of having homogenous Dirichlet conditions, where we
do not need to discuss rigid motions at all we have that for all \( u \in
H^1_0(\Omega) \):
\begin{equation}
    \| \varepsilon(u) \| \geq C\| u \|_1.
\end{equation}

\subsection{Removing The Set Of Rigid Motions}

In order to remove the set of rigid motions, as is required in
\cref{eq:korn_neumann}, we may employ a Lagrange multiplier method. Introducing
the bilinear form \( b \) given as 
\begin{equation}
    b(u, v) = \langle u, v \rangle
\end{equation}
we can formulate a saddle point problem as follows: Find \( u \in H^1(\Omega)
\setminus \mathcal{R} \) and \( \gamma \in \mathcal{R} \) such that 
\begin{align}
    a(u, v) + b(\gamma, v) &= L(v) \text{ for all } v \in H^1_0(\Omega), \\
    b(u, \xi) &= 0 \text{ for all } \xi \in \mathcal{R}.
\end{align}
This is on the same form as \vref{eq:stokes_variational}, and we need to verify
that it complies with the Babuska--Brezzi conditions.

\section{Locking}

The numerical artifact known as \emph{locking} has nothing to do with having
pure Neumann conditions as in the previous section. We therefore discuss this
in the context of homogenous Dirichlet conditions. The linear eliasticity
equation then reads
\begin{align}
    \label{eq:locking_variational}
    -\mu \lapl u - (\mu + \lambda)\grad\div u &= f \text{ in } \Omega, \\
    u &= 0 \text{ on } \Gamma_D.
\end{align}
We have here used the fact that \( \div (\grad u)^T = \grad \div u = \lapl u\)
in the symmetric gradient \( \varepsilon(u) \). Locking is a numerical artifact
that occurs when \( \lambda \gg \mu \), and this is due to the nature of the
operators \( \div \) and \( \grad \). The divergence \( \div \) deals with the
flux through the element edges, hence using vertex based elements gives a bad
approximation, where as the gradient \( \grad \) works fine with vertex based
elements.

\subsection{A Solution for Locking}

The locking phenomenon can be avoided by introducing a solid pressure 
\begin{equation}
    p = (\mu + \lambda) \div u.
\end{equation}
Inserting this in \cref{eq:locking_variational} yields a new system
\begin{align}
    -\mu \lapl u - \grad p &= f \text{ in } \Omega, \\
    \div u - \frac{p}{\mu + \lambda} &= 0 \text{ in } \Omega.
\end{align}
This can be formulated as a system similar to the Stokes problem in a sense
that will be mentioned in the following. Introducing the operators \(a\), \(b\)
and \(c\) as:
\begin{align}
    a(u, v) &= \langle \grad u, \grad v  \rangle, \\
    b(p, v) &= \langle \grad p, v \rangle, \\
    c(p, v) &= \frac{1}{\mu+\lambda} \langle p, q \rangle, 
\end{align}
then we wish to solve the system: Find \( u \in H^1_0(\Omega) \) and \( p \in L^2(\Omega) \)
such that 
\begin{align}
    a(u, v) + b(p, v) &= L(v) \text{ for all } v \in H^1_0(\Omega), \\
    b(u, q) - c(p, q) &= 0 \text{ for all } q \in L^2(\Omega).
\end{align}
In the limit as \( \lambda \to \infty\) we see that the operator \(c \) tends
to zero, \( c \to 0 \). Hence in the limit, this reduces to a Stokes problem,
which we already know is stable in pressure under certain choices for the
finite element pairs.  One might employ a Taylor--Hood element for instance.
