\chapter[Stokes Problem]{Discretization of Stokes Problem}

\begin{problem_text}
Derive a proper variational formulation of the Stokes problem. Discuss the four
Brezzi conditions that are needed for a well-posed continuous problem. Explain
why oscillations might appear in the pressure for some discretization
techniques. Present expected approximation properties for mixed elements that
satisfy the \(\inf\)-\(\sup\) condition, and discuss a few examples like e.g.
Taylor--Hood, Mini, and Crouzeix--Raviart. Discuss also how one might circumvent
the \(\inf\)-\(\sup\) condition by stabilization. 
\end{problem_text}



\section{Finite Element Formulation}

The Stokes problem deals with the flow of incompressible Newtonian fluids
slowly moving in a domain \( \Omega \subseteq \R^n\). The strong formulation of
the problem is given as:
\begin{align}
    -\lapl u + \grad p &= f \text{ in }  \Omega, \\
    \div u &= 0 \text{ in } \Omega,  \\
    u &= g \text{ on } \Gamma_D, \\
    \frac{\partial u}{\partial n} - pn &= h \text{ on } \Gamma_N.
\end{align}
Here \( p : \Omega \to \R \) denotes the fluid pressure, \( u : \Omega \to \R^n
\) denotes the fluid velocity. The body force is denoted \( f \). Note that
this problem has two unknowns, namely \( u \) and \( p \). To this strong
formulation we associate the two bilinear forms \( a \colon \trial \times
\trial \to \R \) and \( b \colon W \times \trial \to \R \) given by
\begin{align}
    a(u, v) &\coloneqq \langle \grad u, \grad v \rangle, \\
    b(p, v) &\coloneqq \langle p, \div v \rangle.
\end{align}
In addition, we also introduce the linear form \( L \colon \trial \to \R \)
given by
\begin{equation}
    L(v) \coloneqq \langle f, v \rangle + \int_{\Gamma_N} h v \, dS.
\end{equation}

\subsection{Weak Formulation}

We can therefore instead consider the weak form of the Stokes problem, namely:
Find \( u \in V \) and \( p \in W \) such that 
\begin{align}
    \label{eq:stokes_variational}
    a(u, v) + b(p, v) &= L(v) \text{ for all } v \in \test,\\
    b(q, u) &= 0 \text{ for all } q \in \hat{W}.
\end{align}

Again we need to precicely state what the trial spaces \( V, W \) and
respective test spaces are. We require \( u \) to equal \( g \) on the
Dirichlet boundary, while \( v \) should vanish on the Dirichlet boundary.
Furthermore, we require only one derivative in \( u \) and \( v \) and no
derivatives in \( p \). We therefore decide on the spaces
\begin{align}
    \trial &\coloneqq H^1_g(\Omega) \subseteq H^1 (\Omega), & \test &\coloneqq
    H^1_0(\Omega) \subseteq H^1(\Omega), \\ 
    W &\coloneqq L^2(\Omega), & \hat{W} &\coloneqq L^2_0(\Omega) \subseteq
    L^2(\Omega).
\end{align}

\subsection{Finite Element Formulation}

In order to compute with these spaces we need to introduce a basis. Assume that
the finite dimensional velocity space \( \trial_h \) is spanned by basis
elements \((\varphi_i)_{i=1}^N \) and that the finite dimensional pressure
space \( W_h \) is spanned by basis elements \((\psi_j)_{j=1}^M\).  Making the
ansatz
\begin{equation}
    u = \sum^{N}_{i=1} c_i \varphi_i, \qquad p = \sum^{M}_{j=1} d_i \psi_j
\end{equation}
we end up with the linear system 
\begin{equation}
    \label{eq:discretized_stokes}
    \begin{bmatrix}
        \A & \B^T \\
        \B & 0
    \end{bmatrix} \begin{bmatrix}
        \c \\
        \d
    \end{bmatrix}
    =
    \begin{bmatrix}
        \b \\
        0
    \end{bmatrix},
\end{equation}
where the matrix elements are given as
\begin{align}
    A_{i, j} &= \langle \grad \varphi_i, \grad \varphi_j \rangle, \\
    B_{i, j} &= \langle \psi_i, \grad \varphi_j \rangle, \\
    b_j &= \langle f, \varphi_j \rangle + \int_{\Gamma_N} h \varphi_j \, dS.
\end{align}

\section{Well Posedness of Weak Formulation}

In the abstract setting, the Stokes problem is a mixed saddle point problem.
There is a set of four conditions --- analogous to the Lax--Milgram theorem
used in the Poisson and Convection-Diffusion problems --- for the Stokes
problem that ensure the existence and uniqueness of the continuous solution.
In the following, we assume that the Dirichlet boundary condition can be
reduced to a homogenous one, in order to be able to apply the Poincar\'e
inequality.

\paragraph{Boundedness of \(a\):}

Applying the Cauchy--Schwartz inequality on \( a \) yields:
\begin{equation}
    a(u, v) = \langle \grad u, \grad v \rangle \leq \|\grad u\|_0 \| \grad v\|_0.
\end{equation}
Noting that the \( L^2\)-norm of the gradient is always smaller than the \( H^1
\) norm, the condition holds. 

\paragraph{Boundedness of \(b \):}

Applying the Cauchy--Schwartz inequality on \( b \) yields:
\begin{equation}
    b(p, v) = \langle p, \div v \rangle \leq \| p \|_0 \|\div v\|_0.
\end{equation}
To show boundedness, it suffices to show that \(\| \div v \|_0 \leq \| v \|_1
\). To this end, note that 
\begin{equation}
    \| \div v \|_0 = \left( \int_\Omega \sum^{n}_{i=1} \left(\frac{\partial
    v_i}{\partial x_i}\right)^2 d\Omega \right)^{1/2}.
\end{equation}
This is merely a subset of the terms occuring in the expression for \( (\grad
u)^2 \), hence we can conclude outright that \( \| \grad v \|_0 \leq \| v\|_1
\), and consequently \(b \) is bounded.

\paragraph{Coercivity of \( a \):}

Start by noting that \(\| u \|_1^2 = \| u \|_0^2 + \| \grad u \|_0^1 \). By the
Poincar\'e inequality, we have that this satisfies
\begin{equation}
    \| u \|_1^2 \leq (C^2 + 1 )|u|_1^2.
\end{equation}
Furthermore, we have that \( |u|_1^2 = a(u, u) \), and hence
\begin{equation}
    a(u, u) \geq \frac{1}{C^2 + 1}\|u\|_1^2
\end{equation}
which shows that \( a \) is indeed coercive.

\paragraph{The \(\inf\)-\(\sup\) condition:}

In order for the discretized Stokes problem to be well posed, we also need to
satisfy the \( \inf \)-\(\sup\) condition, namely that
\begin{equation}
    \label{eq:infsup}
    \sup_{v \in \test} \frac{b(q, v)}{\|v\|_1} > K \| q \|_0 \text{ for all } q \in \hat{W}.
\end{equation}
This can be thought of as the ``coercivity'' of \( b \). This condition ensures
that \( B \) is surjective, or that \( B^T \) is injective. This in turn
ensures that the solution exists and is unique. We will not show that this
holds for the Stokes problem, as it is tricky.

\section{Oscillations in the pressure}

Consider the matrix equations in \vref{eq:discretized_stokes}. Writing these
out yields the system of two equations
\begin{align}
    \A \c + \B^T \d &= \b, \\
    \B\c &= 0.
\end{align}
Recall that \( \c \) are the degrees of freedom for the velocity, and that \(
\d \) are the degrees of freedom for the pressure. Since \( a \rightsquigarrow
\A \) we have that \( \A \) is invertible. Under the assumption that \(\B
\A^{-1} \B^T\) is invertible we can solve for the pressure \( \d \), yielding
\begin{equation}
    \d = (\B\A^{-1}\B^T)^{-1} \B\A^{-1}\b.
\end{equation}
Under what circumstances is \( \B\A^{-1}\B^T \) invertible? Since \( \A \) is
invertible, we need only consider whether \( \B\B^T \) is invertible. This is
equivalent to verifying that \( \Ker(\B) = 0 \), and it is exactly this the \(
\inf\)-\(\sup\) condition in \vref{eq:infsup} gives sufficient conditions for.
Hence if this is not satisfied, we may be solving a non-invertible system when
computing the pressure, which may yield oscillations.

If \( n \) and \( m \) denotes the number of degrees of freedom for the
velocity and the pressure respectively, the block matrix equation is
non-singular if \( n \) is sufficiently large compared to \( m \). This is
because the zero-matrix is of dimension \( m \times m \). Having \( n \gg m\)
ensures that this is comparatively small. 

\section{Error Estimates}

In the following, set \( e_u \coloneqq u - u_h \) and \( e_p \coloneqq p - p_h
\).  For finite element pairs that satisfy the \( \inf \)-\(\sup\) condition
\cref{eq:infsup} we have an error estimate that reads
\begin{equation}
    \| e_u \|_1 + \| e_p \|_0 \leq C h^k \|u\|_{k+1} + D h^{\ell + 1} \|p\|_{\ell + 1}
\end{equation}
where \( k \) and \( \ell \) denotes the polynomial degree of the velocity and
pressure. Finding finite element pairs that satisfies the conditions for this
error estimate is a difficult task. Below are a few examples of such finite
element pairs:
\begin{description}
    \item[The Taylor--Hood element:]
        This element pair consists of a quadratic element for the velocity, and
        a linear element for the pressure. This yields the error estimate
        \begin{equation}
            \| e_u \|_1 + \| e_p \|_1 \leq h^2 \left( C\|u\|_{3} + D \|p\|_{2}\right).
        \end{equation}
    \item[The Crouzeix--Raviart element:]
        This element consists of a linear element in velocity, and a constant
        element in pressure, yielding the error estimate:
        \begin{equation}
            \| e_u \|_1 + \| e_p \|_1 \leq h^1 \left(C \|u\|_{2} + D \|p\|_{1}\right)
        \end{equation}
    \item[The Mini element:]
        This element employs linear elements in both velocity and pressure,
        however the velocity element also contains a cubic bubble in order to
        yield enough degrees of freedom to satisfy the inf-sup condition. We get the error estimate
        \begin{equation}
            \| e_u \|_1 + \| e_p \|_1 \leq C h^1 \|u\|_{2} + D h^{2} \|p\|_{2}
        \end{equation}
\end{description}

\section{Stabilization Techniques}

Instead of solving the system given in \cref{eq:discretized_stokes}, we may
solve an alternative system given as
\begin{align}
    \A\c + \B^T \d &= \b, \\
    \B\c - \varepsilon \D \d &= \varepsilon d.
\end{align}
This alternative, perturbed,  system has coupled the solution of \( \c \) to
the solution of \( \d \) in a way that lets us control the coupling, through
both the matrix \( \D \) which we have yet to define, and the parameter \(
\varepsilon \).  Solving this for the pressure, we get:
\begin{equation}
    \d = (-\varepsilon D)^{-1}(\varepsilon d- \B \c)
\end{equation}
Using this to solve for the velocity \( \c \):
\begin{equation}
    \c = (\d + \D^{-1}d) \left(\A + \frac{1}{\varepsilon}\B^T\D^{-1}\B\right)^{-1}
\end{equation}
provided that the matrix on the right is in fact invertible. It can be verified
that is is by noting that \(\A\) is positive by construction, and \( \B^T
\D^{-1} B \) is positive if \( \D \) is positive. Hence, in choosing \( \D \) we
need to make sure it is in fact positive. Three choices for \( \D \) are
proposed, all based on perturbed versions of the equation of continuity \( \div
u = 0 \):
\begin{enumerate}[(i)]
    \item Setting \( \D \coloneqq \A \), this corresponds to pressure
        stabilization where \(\div v = \varepsilon \lapl p \); 
    \item setting \( \D \coloneqq \boldsymbol{M}\), corresponding to the
        penalty method where \( \div v = \varepsilon p \); or 
    \item setting \( \D \coloneqq (1/\Delta t) M\) corresponding to artificial
        compressibility, \( \div v= -\varepsilon (\partial p/\partial t)
        \).
\end{enumerate}
A problem with such techniques lies in the choice of the parameter \(
\varepsilon \). Choosing \( \varepsilon \) too small, pressure oscillations
occur in the solution. Choosing \( \varepsilon \) too large the accuracy of the
solution deteriorates.
