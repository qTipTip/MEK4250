\documentclass[twocolumn, article]{memoir}

\setlength{\parindent}{0em}
\setlength{\parskip}{0.5cm plus4mm minus3mm}
\usepackage[]{times} 
\usepackage[]{newtxmath} 

\usepackage[]{amsmath} 
\usepackage[]{amsthm} 
\usepackage[]{amssymb} 
\usepackage[]{mathtools} 

\usepackage[]{enumitem} 
\usepackage[]{booktabs} 


\usepackage[]{microtype} 

\usepackage[]{hyperref} 
\usepackage[capitalize, noabbrev]{cleveref} 

\title{\textsc{Mandatory assignment \\ Finite Element Method in Computational
Mechanics \\ MEK4250}}
\author{Ivar Stangeby}
\begin{document}
\maketitle

\chapter*{Exercise 7.2: Stokes problem}

\subsection{Problem statement and strong formulation}    

Let the fluid domain \(\Omega\) be bounded in \(\mathbb{R}^n\) with a
smooth boundary. Denote by \( u \colon \Omega \to \mathbb{R} ^n \) the
fluid velocity, and by \( p \colon \Omega \to \mathbb{R} \) the fluid
pressure. Furthermore, assume that the domain boundary \( \partial\Omega \)
is partitioned into the Dirichlet boundary, and the Neumann boundary,
denoted by \( \partial\Omega_D \) and \( \partial\Omega_N \), respectively.
In its strong form, the Stokes problem reads
\begin{align*}
    -\Delta u + \nabla p &= f, & &\text{in}\quad \Omega;\\
    \nabla \cdot u &= 0, & &\text{in}\quad \Omega;\\
    u &= g, & &\text{on} \quad \partial \Omega_D;\\
    \frac{\partial u}{\partial n} - pn &= h,& &\text{on} \quad \partial\Omega_N.
\end{align*}

\subsection{Well posedness}
\label{sec:well_posedness}

We wish to show the well posedness of the weak formulation of Stokes problem.
In our case, we deal with the sobolev space \( H^1_0(\Omega) \) for the fluid \( u \),
and the \( L^2(\Omega) \) space for the pressure \( p \).  This, involves showing that
the following conditions are met:
\begin{enumerate}[wide, label=\roman*)]
    \item \begin{equation*}
            a(u, v) \leq C_1 \|u\|_{H^1} \|v\|_{L^2},
    \end{equation*}
    \item \begin{equation*}
            b(p, v) \leq C_2 \|p\|_{L^2} \|v\|_{H^1},
    \end{equation*}
    \item \begin{equation*}
            a(u, u) \geq C_3 \|u\|_{H^1}^2,
    \end{equation*}
\end{enumerate}
for all \( u, v \in H^1_0(\Omega) \) and all \( p \in L^2 \).
Recall that the bilinear forms in question are given by
\begin{align*}
    a(u, v) &= \int_\Omega \nabla u : \nabla v \, \mathrm{d}x,\\
    b(p, v) &= \int_\Omega p \nabla \cdot v \, \mathrm{d}x
\end{align*}
where \( \boldsymbol{A} : \boldsymbol{B} = \sum_{i,j} A_{ij}B_{ij} \) denotes
the Frobenius inner product.

\begin{enumerate}[wide, label=\roman*)]
    \item Applying the Cauchy-Schwartz inequality on \( a \) yields the following:
\begin{align*}
    a(u, v) &= \int_\Omega \nabla u : \nabla v \, \mathrm{d}x \\
            &= \langle \nabla u \mid \nabla v \rangle \leq |u|_{H^1}|v|_{H^1}.
\end{align*}
Noting that the seminorm on \( H^1_0 \)is never larger than the full norm on \(
H^1_0\), so the condition holds. 
\item Applying Cauchy-Schwartz on \( b \) yields
\begin{equation*}
    b(p, v) \leq \| p \|_{L^2} \| \nabla \cdot v\|_{L^2}.
\end{equation*}
To this end, it suffices to show that \( \|\nabla \cdot v \|_{L^2} \leq \| v
\|_{H^1} \). For the left hand side, we have that
\begin{equation*}
    \|\nabla \cdot v \|_{L^2} \leq \left( \int_\Omega \sum^{n}_{i=1} \left( \frac{\partial v_i}{\partial x_i} \right)^2 \right)^{1/2}.
\end{equation*}
For the right hand side, firstly we have that
\((\nabla v)^2 = \sum^{n}_{i=1} \sum^{n}_{j=1} \left( \partial v_j / \partial
x_i\right)^2 \). Note that this has a striking resemblance to \( \nabla \cdot v
\), however, with a lot more positive terms. We can therefore conclude outright that
\begin{equation*}
    b(p, v) \leq \|p\|_{L^2}\|v\|_{H^1}.
\end{equation*}
\item Finally, we consider \(a(u, u)\). Firstly, we have that \(\| u \|_{H^1}^2 = \|u
\|_{L^2}^2 + |u|_{H_1}^2\). By applying Poincar\'e, we know that this has to be
less than or equal to \((C^2 + 1)|u|_{H^1}^2\). Writing this out, we have that
\begin{equation*}
    \|u\|^2_{H^1} \leq (C^2 + 1)\int_\Omega (\nabla u)^2 \, \mathrm{d}x.
\end{equation*}
Note that \( (\nabla u)^2 = \nabla u : \nabla u \). Indeed, 
\begin{equation*}
    (\nabla u)^2 = \sum_{j=1}^n \sum_{i=1}^n \left(\frac{\partial u_i}{\partial x_j}\right)^2
    \, = \nabla u : \nabla u.
\end{equation*}
Consequently, we have that \(\|u\|^2_{H^1} \leq (C^2 + 1) a(u, u)\).
Multiplying each side of the equation by \(D \coloneqq 1 / (C^2 + 1)\) we get
the bound we wanted.
\end{enumerate}

\chapter*{Exercise 7.6: Approximation order in Stokes problem}
\label{cha:exercise_7_6_approximation_order_in_poiseuille_flow}

In this exercise we solve Stokes problem as above, with the known solutions:
\begin{align*}
    u &= (\sin(\pi y), \cos(\pi x)), \\
    p &= -\sin(2 \pi x).
\end{align*}
It then follows that the source term is given as:
\begin{equation}
    \notag
    f(x) = (\pi^2 \sin(\pi y) - 2 \pi \cos(2 \pi x), \pi^2 \cos(\pi x)).
\end{equation}
Ideally, we obtain optimal convergence rate, given by
\begin{equation}
    \notag
    \|u - u_h\|_{H^1} + \| p - p_h\|_{L^2} \leq Ch^k\|u\|_{H^{k+1}} + Dh^{\ell + 1} \| p \|_{H^{\ell + 1}},
\end{equation}
where \( k \) and \( \ell \) are the polynomial degree of the velocity and
pressure.  In order to obtain this optimal convergence, we need to determine
the finite element pairs \( V_h \) and \( Q_h\) for the velocity and pressure,
respectively, that satisfy the Babuska--Brezzi condition:
\begin{equation}
    \label{eq:bb_cond}
    \sup_{v_h \in V_{h, g}} \frac{(p_h, \nabla \cdot v_h)}{\|v_h\|_{H^1}} \geq \beta
    \|p_h\|_{L^{2}} > 0, 
\end{equation}
for all \(p_h \in Q_h\).
Elements satisfying \cref{eq:bb_cond} include the following:
\begin{description}
    \item[Taylor-Hood:]
        Quadratic for velocity, and linear for the pressure.
    \item[Crouzeix-Raviart:]
        Linear in velocity, constant in pressure.
    \item[Mini element:]
        Linear in both velocity and pressure, and a cubic bubble function is
        added to the velocity element in order to satisfy \cref{eq:bb_cond}.
\end{description}

In this exercise, we wish to examine whether the approximation is of the
expected order for the \(P_4\) -- \(P_3\), \(P_4\)--\(P_2\), \(P_3\) --
\(P_2\), and \(P_3\) -- \(P_1\) elements. We examine these in turn.  If the
convergence is optimal, we would expect the following:
\begin{description}[wide]
    \item[\(P_4\)--\(P_3\):] 
    With \( k = 4 \) and \(\ell = 3\), we would expect the error to run as \(
    \mathcal{O}(h^4) \).

    \item[\(P_4\)--\(P_2\):]
    With \( k = 4 \) and \(\ell = 2\), we would expect the error to run as \(
    \mathcal{O}(h^3) \).

    \item[\(P_3\)--\(P_2\):]
    With \( k = 3 \) and \(\ell = 2\), we would expect the error to run as \(
    \mathcal{O}(h^3) \).

    \item[\(P_3\)--\(P_1\):]
    With \( k = 3 \) and \(\ell = 2\), we would expect the error to run as \(
    \mathcal{O}(h^2) \).
\end{description}
This is verified by computations, as can be seen in
\cref{tab:approximation_order_7_6}. 

\begin{table}[htpb]
    \centering
    \caption{The convergence rate of the error is given as \( \alpha \), and
        the error constant is \( C \) in the error estimate \[ \| u -
        u_h\|_{H^1} + \| p - p_h \|_{L^2} \leq Ch^\alpha.\] These were computed
        using a linear polynomial fit of the computed errors as \( h \)
        decrease. The convergence rates seem to agree with predictions.}
    \label{tab:approximation_order_7_6}
    \begin{tabular}{lcc}
    \toprule
    {Element} & \( \alpha \) & \(C\) \\
    \midrule
    \(P_4\) -- \(P_3\) &          4.04615 &    0.344373 \\
    \(P_4\) -- \(P_2\) &          2.92402 &     1.47968 \\
    \(P_3\) -- \(P_2\) &          2.91623 &      1.3616 \\
    \(P_3\) -- \(P_1\) &          2.02219 &     2.26125 \\
    \bottomrule
    \end{tabular}
\end{table}
\end{document}
