\documentclass[]{article}

\usepackage[utf8]{inputenc}
\usepackage[]{palatino}
\usepackage[]{eulervm}
\usepackage[]{amsmath, amsthm, amssymb} 
\usepackage[]{thmtools} 
\usepackage[]{mdframed} 
\usepackage[]{booktabs} 
\usepackage[]{subcaption} 
\usepackage[margin=2in]{geometry} 

\usepackage[]{hyperref} 
\usepackage[capitalize, noabbrev]{cleveref} 

\renewcommand{\texttt}[1]{\textcolor{Maroon}{#1}}
\usepackage[usenames, dvipsnames, svgnames, table]{xcolor} 
\declaretheoremstyle[
    headfont = \color{Maroon}\normalfont\bfseries,
    leftmargin=-10t,
    mdframed = {
        linecolor=Gray,
        backgroundcolor=Gray!20
    }
]{BVP}
\declaretheorem[
    style=BVP,
    numbered=yes,
    name=Boundary Value Problem,
]{BVP}

\usepackage[]{mathtools} 

\newcommand{\lapl}{\Delta}
\newcommand{\norm}[2]{\| #1 \|_{#2}}
\newcommand{\x}{\mathbf{x}}
\renewcommand{\u}{\mathbf{u}}

\title{\textsc{Assignment 1 \\ The Finite element method in computational mechanics}}
\author{Ivar Haugaløkken Stangeby}

\begin{document}
\maketitle    
\section*{Exercise 1}
\label{sec:formulation_of_problem}

In this assignment we start off by considering the following boundary value
problem.

\begin{BVP}
    \label{bvp:one}
    On the two dimensional domain $\Omega \coloneqq (0, 1)^2$, consider the
    problem:
    \begin{align}
        \label{eq:problem}
        -\nabla u &= f \text{ in } \Omega,\\ 
        u &= 0 \text{ for } x = 0 \text{ and } x = 1,\\
        \frac{\partial u}{\partial n} &= 0 \text{ for } y = 0 \text{ and } y = 1.
    \end{align}
\end{BVP}

\subsection*{Analytical gobbledygook}
\label{sub:analytical_}


We start by assuming $u = \sin(\pi k x)\cos(\pi k y)$ and compute the source
term $f = -\lapl u = 2\pi^2 k^2 u$.
We wish to compute analytically, the $H^p$ norm.  Recall that the $H^p$  norm
$\| \cdot \|_p$ is defined by
\begin{equation}
    \notag
    \| u \|_p = \left( \sum^{}_{|\alpha|\leq p} \int_\Omega
    \big(\frac{\partial^{|\alpha|} u}{\partial \x^\alpha}\big)^2 \, d\x
\right)^{1/2}
\end{equation}
where $\alpha \coloneqq (\alpha_1, \ldots, \alpha_d)$ is a multi-index, and
$|\alpha| \coloneqq \alpha_1 + \ldots + \alpha_d$. In the case where $\Omega$
is a subset of $\mathbb{R}^2$, we have $\alpha = (i, j)$ and $|\alpha| = i +
j$. Note that the terms in the sum occur as the $L^2$ norm squared of the mixed
partial derivatives, i.e.,
\begin{align*}
    \notag
    \left\| \frac{\partial^{i+j}u}{\partial x^i \partial y^j} \right\|_{L^2}^2
    = (k \pi)^{2(i+j)}\int_0^1\int_0^1 \cos^2(k\pi x) \sin^2(k\pi y) \, dx dy.
\end{align*}
Using the fact that both $\sin^2(\pi k y)$ and $\cos^2(\pi k x)$ integrate to
$1/2$ over the unit interval, we have that this equals $(k \pi)^{2(i+j)}/4$.
For $|\alpha| = n$, we have $n + 1$ partial derivatives of order $n$, hence
$\| u\|_p$ can be computed as
\begin{equation}
    \notag
    \| u \|_p = \frac{1}{2}\sum^{}_{|\alpha| \leq p} \sum_{r = 0}^{ |\alpha| } (\pi k)^{i+j}.
\end{equation}

\subsection*{Numerical error estimates}
\label{sub:simulations}

We solve the system given in \cref{bvp:one} in the \textsc{Python}-framework
\texttt{FeniCS}. Our mesh is taken to be uniformly spaced with mesh size $h
\coloneqq 1 / N$. We examine the error in both the $L_2$ and the $H^1$ norms
for $k = 1, 10$ and for both first and second order Lagrangian elements, that is
\begin{align*}
    \text{error} = \| u - u_h \|_q && \text{ for } q = 0, 1.
\end{align*}
The numerical errors are listed in \cref{tbl:errors_1}. 

We now wish to verify the two following error estimates:
\begin{align}
    \label{eq:errors_one}
    \|u - u_h\|_1 \leq C_\alpha h^\alpha,
    \shortintertext{and}
    \|u - u_h\|_0 \leq C_\beta h^\beta.
\end{align}
These error estimates can be rewritten as linear equations in $h$ with slopes
$\alpha$, $\beta$ and constant terms $\log(C_\alpha)$, $\log(C_\beta)$,
respectively. That is
\begin{equation}
    \notag
    \log(\|u - u_h\|_1) \leq \alpha h + \log(C_\alpha),
\end{equation}
and similarly for the $\| \cdot \|_0$ error. Sampling the left hand side for
several values of $h$, we can fit a linear function to the data, hence finding
the unknown slope and constant terms. This has been done using the function
\texttt{numpy.polyfit()}. The results are given in \cref{tbl:polyfit_one}.

Plotting the errors against the number of elements $N$ in a $\log$-$\log$ plot
reveals a linear tendency. 

\begin{table}[htpb]
    \centering

    \caption{The $L_2$ and $H^1$ errors for varying number of mesh-elements.
    First order elements to the left and second order elements to the right.}

    \label{tbl:errors_1}

    \subcaption{Errors for $k = 1$.}
    \begin{tabular}{ccc}
        \toprule
        $N$  & $L_2$ & $H^1$\\
        \midrule
        8	& 0.62583&	 3.03866 \\
        16	& 0.64926&	 3.11820 \\
        32	& 0.65536&	 3.13876 \\
        64	& 0.65690&	 3.14394 \\
        \bottomrule
    \end{tabular}\hspace{4em}
    \begin{tabular}{ccc}
        \toprule
        $N$  & $L_2$ & $H^1$\\
        \midrule
        8	&0.65310	&3.11978\\
        16	&0.65637	&3.13927\\
        32	&0.65716	&3.14408\\
        64	&0.65735	&3.14528\\
        \bottomrule
    \end{tabular}\\[2em]

    \subcaption{Errors for $k = 10$.}

    \begin{tabular}{ccc}
        \toprule
        $N$  & $L_2$ & $H^1$\\
        \midrule
        8	 &0.74298	 &24.94148\\
        16	 &0.50061	 &28.59929\\
        32	 &0.76293	 &37.48450\\
        64	 &0.91167	 &41.68281\\
        \bottomrule
    \end{tabular}\hspace{4em}
    \begin{tabular}{ccc}
        \toprule
        $N$  & $L_2$ & $H^1$\\
        \midrule
        8	&0.65808	&26.95569\\
        16	&0.71521	&32.92179\\
        32	&0.90010	&40.14219\\
        64	&0.95659	&42.52210\\
        \bottomrule
    \end{tabular}

\end{table}
\begin{table}[htpb]
    \centering
    \caption{The slopes and coefficients for the error estimates given in
    \cref{eq:errors_one}.}
    \label{tbl:polyfit_one}
    \begin{tabular}{ccccc}
        \toprule
        $k$ & $C_\alpha$ & $\alpha$ & $C_\beta$ & $\beta$\\
        \midrule
        1 & & & & \\
        10 & & & & \\
        \bottomrule
    \end{tabular}
\end{table}

\section*{Exercise 2}
\label{sec:exercise_2}

We now consider another boundary value problem:
\begin{BVP}
    \label{bvp:two}
    On the two dimensional domain $\Omega \coloneqq (0, 1)^2$, consider the
    second order problem:
    \begin{align}
        \label{eq:problem_2}
        -\mu\Delta u + u_x &= 0 \text{ in } \Omega,\\ 
        u &= 0 \text{ for } x = 0,\label{eq:d_one} \\
        u &= 1 \text{ for } x = 1,\label{eq:d_two}\\
        \frac{\partial u}{\partial n} &= 0 \text{ for } y = 0 \text{ and } y = 1.\label{eq:neumann}
    \end{align}
\end{BVP}

\subsection*{Analytical solution}
\label{sec:analytical_solution}

It is possible to derive an analytical solution for the above boundary value
problem using seperation of variables. We make the ansatz that we can write
$u(x, y) = f(x)g(y)$. Plugging this into \cref{bvp:two}, and dividing by $-\mu
u$ we arrive at the set of equations
\begin{align}
    f''(x) - \frac{1}{\mu}f'(x) - Cf(x) &= 0, \label{eq:f}\\
    g''(y) + Cg(y) &= 0,
\end{align}
where $C$ is some unknown constant. Solving for $g(y)$ first, we arrive at
the solution
\begin{equation}
    \notag
    g(y) = A\sin(\sqrt{C}y) + B\cos(\sqrt{C}y).
\end{equation}
Enforcing the Neumann boundary conditions given in \cref{eq:neumann}, we
determine $g$ to be constant (with respect to $x$) equal to 
\begin{equation}
    \notag
    g(y) = B\cos(n\pi y), 
\end{equation}
with $n \in \mathbb{N}$. In particular, for $n = 0$, $C = 0$ so we have $g(y) = B$. Furthermore, with this choice of $n$, \cref{eq:f} reduces to
\begin{equation}
    \notag
    f''(x) - \frac{1}{\mu}f'(x) = 0
\end{equation}
which has solution $ f(x) = D e^{\frac{1}{\mu}x} + E$.  Enforcing the Dirichlet
boundary conditions given in \cref{eq:d_one,eq:d_two} we determine $E = -1$ and
$D = (e^{\frac{1}{\mu}} - 1)^{-1}$, as well as $B = 1$, yielding the final
solution
\begin{equation}
    \notag
    u(x, y) = f(x)g(y) = \frac{e^{\frac{1}{\mu}x} - 1}{e^{\frac{1}{\mu}} - 1}.
\end{equation}
\end{document}
