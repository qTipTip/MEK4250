\documentclass[]{article}

\usepackage[utf8]{inputenc}
\usepackage[]{palatino}
\usepackage[]{eulervm}
\usepackage[]{amsmath, amsthm, amssymb} 
\usepackage[]{thmtools} 
\usepackage[]{mdframed} 
\usepackage[]{booktabs} 
\usepackage[]{subcaption} 
\usepackage[margin=2in]{geometry} 

\usepackage[]{hyperref} 
\usepackage[capitalize, noabbrev]{cleveref} 

\renewcommand{\texttt}[1]{\textcolor{Maroon}{#1}}
\usepackage[usenames, dvipsnames, svgnames, table]{xcolor} 

\declaretheoremstyle[
    headfont = \color{Maroon}\normalfont\bfseries,
    leftmargin=-10t,
    mdframed = {
        linecolor=Gray,
        backgroundcolor=Gray!10
    }
]{BVP}
\declaretheorem[
    style=BVP,
    numbered=yes,
    name=Boundary Value Problem,
]{BVP}

\usepackage[]{mathtools} 

\newcommand{\lapl}{\Delta}
\newcommand{\norm}[2]{\| #1 \|_{#2}}
\newcommand{\x}{\mathbf{x}}
\renewcommand{\u}{\mathbf{u}}

\title{\textsc{Assignment 1 \\ The Finite element method in computational mechanics}}
\author{Ivar Haugaløkken Stangeby}

\begin{document}
\maketitle    
\section*{Exercise 1}
\label{sec:formulation_of_problem}

In this assignment we start off by considering the following boundary value
problem.

\begin{BVP}
    \label{bvp:one}
    On the two dimensional rectangular domain $\Omega \coloneqq (0, 1)^2$, consider the
    problem:
    \begin{align}
        \label{eq:problem}
        -\nabla u &= f \text{ in } \Omega,\\ 
        u &= 0 \text{ for } x = 0 \text{ and } x = 1,\\
        \frac{\partial u}{\partial n} &= 0 \text{ for } y = 0 \text{ and } y = 1.
    \end{align}
\end{BVP}

\subsection*{Analytical gobbledygook}
\label{sub:analytical_}


We start by assuming $u = \sin(\pi k x)\cos(\pi k y)$ and compute the source
term $f = -\lapl u = 2\pi^2 k^2 u$.
We wish to compute analytically, the $H^p$ norm.  Recall that the $H^p$  norm
$\| \cdot \|_p$ is defined by
\begin{equation}
    \notag
    \| u \|_p = \left( \sum^{}_{|\alpha|\leq p} \int_\Omega
    \big(\frac{\partial^{|\alpha|} u}{\partial \x^\alpha}\big)^2 \, d\x
\right)^{1/2}
\end{equation}
where $\alpha \coloneqq (\alpha_1, \ldots, \alpha_d)$ is a multi-index, and
$|\alpha| \coloneqq \alpha_1 + \ldots + \alpha_d$. In the case where $\Omega$
is a subset of $\mathbb{R}^2$, we have $\alpha = (i, j)$ and $|\alpha| = i +
j$. Note that the terms in the sum occur as the $L^2$ norm squared of the mixed
partial derivatives, for instance:
\begin{align*}
    \notag
    \left\| \frac{\partial^{i+j}u}{\partial x^i \partial y^j} \right\|_{L^2}^2
    = (k \pi)^{2(i+j)}\int_0^1\int_0^1 \cos^2(k\pi x) \sin^2(k\pi y) \, dx dy.
\end{align*}
Using the fact that both $\sin^2(\pi k y)$ and $\cos^2(\pi k x)$ integrate to
$1/2$ over the unit interval, we have that this equals $(k \pi)^{2(i+j)}/4$.
For $|\alpha| = n$, we have $n + 1$ partial derivatives of order $n$, hence
$\| u\|_p$ can be computed as
\begin{equation}
    \notag
    \| u \|_p = \frac{1}{2} 
    \Big(\sum^{}_{|\alpha| \leq p} (k\pi)^{2|\alpha|}\Big)^{1/2}.
\end{equation}

\subsection*{Numerical error estimates}
\label{sub:simulations}

We solve the system given in \cref{bvp:one} in the \textsc{Python}-framework
\texttt{FeniCS}. Our mesh is taken to be uniformly spaced with mesh size $h
\coloneqq 1 / N$. We examine the error in both the $L_2$ and the $H^1$ norms
for $k = 1, 10$ and for both first and second order Lagrangian elements, that is
\begin{align*}
    \text{error} = \| u - u_h \|_q && \text{ for } q = 0, 1.
\end{align*}
Computed by the function \texttt{exercise\_1\_b()}, the numerical errors are
listed in \cref{tbl:errors_1}. 

We now wish to verify the two following error estimates:
\begin{align}
    \|u - u_h\|_1 \leq C_\alpha h^\alpha,\label{eq:errors_one}
    \shortintertext{and}
    \|u - u_h\|_0 \leq C_\beta h^\beta.
\end{align}
These error estimates can be rewritten as linear equations in $h$ with slopes
$\alpha$, $\beta$ and constant terms $\log(C_\alpha)$, $\log(C_\beta)$,
respectively. That is
\begin{equation}
    \notag
    \log(\|u - u_h\|_1) \leq \alpha h + \log(C_\alpha),
\end{equation}
and similarly for the $\| \cdot \|_0$ error. Sampling the left hand side for
several values of $h$, we can fit a linear function to the data, hence finding
the unknown slope and constant terms. This has been done using the function
\texttt{numpy.polyfit()}, and the full implementation can be seen in
\texttt{estimate\_error()}. The results are given in \cref{tbl:polyfit_one}.

\begin{table}[htpb]
    \centering

    \caption{The $L_2$ and $H^1$ errors for varying mesh size,
    using both first order and second order Lagrangian elements.}

    \label{tbl:errors_1}

    \subcaption{$L_2$ error to the left, $H^1$ error to the right. Using first
    order elements.}
    \begin{tabular}{ccc}
        \toprule
        {$N$} &        $k=$1  &        $k=10$ \\
        \midrule
        8  &  0.032766 &  0.677496 \\
        16 &  0.008462 &  0.363384 \\
        32 &  0.002133 &  0.177866 \\
        64 &  0.000534 &  0.054880 \\
        \bottomrule
    \end{tabular}\hspace{4em}
    \begin{tabular}{ccc}
        \toprule
        {$N$} &       $k=1$  &       $k=10$ \\
        \midrule
        8  &  0.436116 &  25.511516 \\
		16 &  0.218105 &  17.233579 \\
		32 &  0.109047 &  10.543850 \\
		64 &  0.054523 &   5.430879 \\
        \bottomrule
    \end{tabular}\vspace{2em}

    \subcaption{$L_2$ error to the left, $H^1$ error to the right. Using second
    order elements.}
    
    \begin{tabular}{ccc}
        \toprule
        {$N$} &        $k=1$  &        $k=10$ \\
        \midrule
        8  &  0.000569 &  0.424446 \\
        16 &  0.000069 &  0.088649 \\
        32 &  0.000009 &  0.010174 \\
        64 &  0.000001 &  0.001139 \\
        \bottomrule
    \end{tabular}\hspace{4em}
    \begin{tabular}{ccc}
        \toprule
        {$N$} &       $k=1$  &       $k=10$ \\
        \midrule
        8  &  0.033141 &  17.666883 \\
        16 &  0.008387 &   6.717151 \\
        32 &  0.002105 &   1.961954 \\
        64 &  0.000527 &   0.517347 \\
        \bottomrule
    \end{tabular}

\end{table}
\begin{table}[htpb]
    \centering
    \caption{The slopes and coefficients for the error estimates given in
    \cref{eq:errors_one}, for both first and second order elements.}
    \label{tbl:polyfit_one}
    \subcaption{First order elements.}
    \begin{tabular}{ccccc}
        \toprule
        {$k$} &      $\alpha$ &   $C_\alpha$ &         $\beta$ &   $C_\beta$ \\
        \midrule
		1  &  1.980242 &  2.029814 &  1.991671 &    2.090703 \\
        10 &  1.190830 &  9.085196 &  1.705687 &  677.310236 \\
        \bottomrule
    \end{tabular}\vspace{2em}

    \subcaption{Second order elements.}
    \begin{tabular}{ccccc}
        \toprule
        {$k$} &        $\alpha$ &   $C_\alpha$ &         $\beta$ &   $C_\beta$ \\
        \midrule
        1  &  1.980242 &  2.029814 &  0.999942 &    3.488758 \\
        10 &  1.190830 &  9.085196 &  0.740449 &  126.848072 \\
        \bottomrule
\end{tabular}

\end{table}

\section*{Exercise 2}
\label{sec:exercise_2}

We now consider another boundary value problem:
\begin{BVP}
    \label{bvp:two}
    On the two dimensional rectangular domain $\Omega \coloneqq (0, 1)^2$, consider the
    second order problem:
    \begin{align}
        \label{eq:problem_2}
        -\mu\Delta u + u_x &= 0 \text{ in } \Omega,\\ 
        u &= 0 \text{ for } x = 0,\label{eq:d_one} \\
        u &= 1 \text{ for } x = 1,\label{eq:d_two}\\
        \frac{\partial u}{\partial n} &= 0 \text{ for } y = 0 \text{ and } y = 1.\label{eq:neumann}
    \end{align}
\end{BVP}

\subsection*{Analytical solution}
\label{sec:analytical_solution}

It is possible to derive an analytical solution for the above boundary value
problem using seperation of variables. We make the ansatz that we can write
$u(x, y) = f(x)g(y)$. Plugging this into \cref{bvp:two}, and dividing by $-\mu
u$ we arrive at the set of equations
\begin{align}
    f''(x) - \frac{1}{\mu}f'(x) - Cf(x) &= 0, \label{eq:f}\\
    g''(y) + Cg(y) &= 0,
\end{align}
where $C$ is some unknown constant. Solving for $g(y)$ first, we arrive at
the solution
\begin{equation}
    \notag
    g(y) = A\sin(\sqrt{C}y) + B\cos(\sqrt{C}y).
\end{equation}
Enforcing the Neumann boundary conditions given in \cref{eq:neumann}, we
determine $g$ to be constant (with respect to $x$) equal to 
\begin{equation}
    \notag
    g(y) = B\cos(n\pi y), 
\end{equation}
with $n \in \mathbb{N}$. In particular, for $n = 0$, $C = 0$ so we have $g(y) =
B$. Furthermore, with this choice of $n$, \cref{eq:f} reduces to
\begin{equation}
    \notag
    f''(x) - \frac{1}{\mu}f'(x) = 0
\end{equation}
which has solution $ f(x) = D \exp(\frac{1}{\mu}x) + E$.  Enforcing the Dirichlet
boundary conditions given in \cref{eq:d_one,eq:d_two} we determine $E = -1$ and
$D = (e^{\frac{1}{\mu}} - 1)^{-1}$, as well as $B = 1$, yielding the final
solution
\begin{equation}
    \notag
    u(x, y) = f(x)g(y) = \frac{e^{\frac{1}{\mu}x} - 1}{e^{\frac{1}{\mu}} - 1}.
\end{equation}

\subsection*{Streamline diffusion / Petrov-Galerkin.}
\label{sub:streamline_diffusion_petrov_galerkin_}

The parameter $\mu$ causes problems for low values of $\mu$, as later can be
seen in the numerical errors. This can be remedied by introducing the test
function
\begin{equation}
    \notag
    w = v + \beta \nabla v.
\end{equation}
Setting $\beta = h / 2$, where $h$ is the size of the discretized mesh,
corresponds to the upwinding scheme.

\subsection*{Numerical error estimates}

Again, we look at the numerical errors, both $L_2$ and $H^1$ using both first
and second order Lagrange elements. We first examine what happens with the
regular weak formulation, and then we see how the SUPG-method may improve our
results. 


\paragraph{Without SUPG:} Calling \texttt{exercise\_t\_b()} with the
\texttt{SUPG}-flag set to false, we achieve the errors listed in
\cref{tbl:errors_no_supg}. In all cases we see that the error convergence is
slower for both norms for lower values of $\mu$.  Similarily to what was done
in the previous boundary value problem, we estimate the values for $C_\alpha$,
$C_\beta$, $\alpha$ and $\beta$, using \texttt{estimate\_error()}. The results
can be seen in \cref{tbl:error_estimates_no_supg}.

\paragraph{With SUPG:} Setting the SUPG-flag to true, we achieve the values
listed in \cref{tbl:errors_supg} and \cref{tbl:error_estimates_supg}. As we see
from the values, the SUPG-method is superior for low values of $\mu$ and
coarse meshes.

\begin{table}[htpb]
    \centering
    \caption{The $L_2$ and $H^1$ errors of the new boundary value problem,
    using both first and second order elements for varying values of $\mu$.}
    \label{tbl:errors_no_supg}
    
    \subcaption{$L_2$ error using first order elements.}
    \begin{tabular}{cccc}
        \toprule
        {$N$} &         $\mu = 1.00$ &         $\mu = 0.10$ &       $\mu=0.01$ \\
        \midrule
        8  &  0.001402 &  0.023747 &  0.238965 \\
        16 &  0.000351 &  0.006177 &  0.103990 \\
        32 &  0.000088 &  0.001561 &  0.038142 \\
        64 &  0.000022 &  0.000391 &  0.011255 \\
        \bottomrule
    \end{tabular}\vspace{2em}

    \subcaption{$H^1$ error using first order elements.}
    \begin{tabular}{cccc}
        \toprule
        {$N$} &         $\mu = 1.00$ &         $\mu = 0.10$ &       $\mu=0.01$ \\
        \midrule
        8  &  0.037522 &  0.769237 &  7.796998 \\
        16 &  0.018766 &  0.398389 &  7.008644 \\
        32 &  0.009383 &  0.201077 &  5.086480 \\
        64 &  0.004692 &  0.100781 &  2.982329 \\
        \bottomrule
    \end{tabular}\vspace{2em}

    \subcaption{$L_2$ error using second order elements.}
    \begin{tabular}{cccc}
        \toprule
        {$N$} &         $\mu = 1.00$ &         $\mu = 0.10$ &       $\mu=0.01$ \\
        \midrule
        8  &  0.000012 &  0.002248 &  0.086719 \\
        16 &  0.000001 &  0.000304 &  0.030833 \\
        32 &  0.000000 &  0.000039 &  0.007649 \\
        64 &  0.000000 &  0.000005 &  0.001329 \\
        \bottomrule
    \end{tabular}\vspace{2em}

    \subcaption{$H^1$ error using second order elements.}
    \begin{tabular}{cccc}
        \toprule
        {$N$} &         $\mu = 1.00$ &         $\mu = 0.10$ &       $\mu=0.01$ \\
        \midrule
        8  &  0.000597 &  0.118721 &  5.633591 \\
        16 &  0.000150 &  0.031667 &  3.801062 \\
        32 &  0.000038 &  0.008068 &  1.736641 \\
        64 &  0.000009 &  0.002028 &  0.569069 \\
        \bottomrule
    \end{tabular}
\end{table}

\begin{table}[htpb]
    \centering
    \caption{Error estimates for both first and second order Lagrangian
    elements, for varying values of $\mu$.}
    \label{tbl:error_estimates_no_supg}
    
    \subcaption{First order elements.}
    \begin{tabular}{ccccc}
        \toprule
        {$\mu$} &    $\alpha$ &    $C_\alpha$ &      $\beta$ &    $C_\beta$ \\
        \midrule
        1.00 &  1.999763 &  0.089724 &  0.999856 &   0.300108 \\
        0.10 &  1.975224 &  1.458317 &  0.978303 &   5.936369 \\
        0.01 &  1.467166 &  5.552503 &  0.462191 &  22.684702 \\
        \bottomrule
    \end{tabular}\vspace{2em}
    \subcaption{Second order elements.}
    \begin{tabular}{ccccc}
        \toprule
        {$\mu$} &    $\alpha$ &    $C_\alpha$ &      $\beta$ &    $C_\beta$ \\
        \midrule
        1.00 &  2.994029 &  0.005828 &  1.994016 &   0.037794 \\
        0.10 &  2.950373 &  1.059090 &  1.958612 &   7.086985 \\
        0.01 &  2.009556 &  6.772810 &  1.105224 &  67.385481 \\
        \bottomrule
    \end{tabular}
\end{table}

\begin{table}[htpb]
    \centering
    \caption{The $L_2$ and $H^1$ errors of the new boundary value problem,
    using both first and second order elements for varying values of $\mu$, using the SUPG-method.}
    \label{tbl:errors_supg}
    
    \subcaption{$L_2$ error using first order elements.}
    \begin{tabular}{cccc}
        \toprule
        {$N$} &         $\mu = 1.00$ &         $\mu = 0.10$ &       $\mu=0.01$ \\
        \midrule
        8  &  0.008173 &  0.116845 &  0.200561 \\
        16 &  0.003970 &  0.063322 &  0.131785 \\
        32 &  0.001956 &  0.033130 &  0.079917 \\
        64 &  0.000971 &  0.016982 &  0.045471 \\
        \bottomrule
    \end{tabular}\vspace{2em}

    \subcaption{$H^1$ error using first order elements.}
    \begin{tabular}{cccc}
        \toprule
        {$N$} &         $\mu = 1.00$ &         $\mu = 0.10$ &       $\mu=0.01$ \\
        \midrule
        8  &  0.044519 &  1.008406 &  5.434785 \\
        16 &  0.022577 &  0.622238 &  5.792799 \\
        32 &  0.011370 &  0.351801 &  4.973273 \\
        64 &  0.005705 &  0.188202 &  3.626298 \\
        \bottomrule
    \end{tabular}\vspace{2em}

    \subcaption{$L_2$ error using second order elements.}
    \begin{tabular}{cccc}
        \toprule
        {$N$} &         $\mu = 1.00$ &         $\mu = 0.10$ &       $\mu=0.01$ \\
        \midrule
        8  &  0.008173 &  0.116845 &  0.200561 \\
        16 &  0.003970 &  0.063322 &  0.131785 \\
        32 &  0.001956 &  0.033130 &  0.079917 \\
        64 &  0.000971 &  0.016982 &  0.045471 \\
        \bottomrule
    \end{tabular}\vspace{2em}

    \subcaption{$H^1$ error using second order elements.}
    \begin{tabular}{cccc}
        \toprule
        {$N$} &         $\mu = 1.00$ &         $\mu = 0.10$ &       $\mu=0.01$ \\
        \midrule
        8  &  0.044519 &  1.008406 &  5.434785 \\
        16 &  0.022577 &  0.622238 &  5.792799 \\
        32 &  0.011370 &  0.351801 &  4.973273 \\
        64 &  0.005705 &  0.188202 &  3.626298 \\
        \bottomrule
    \end{tabular}
\end{table}

\begin{table}[htpb]
    \centering
    \caption{Error estimates for both first and second order Lagrangian
    elements, for varying values of $\mu$, using the SUPG-method.}
    \label{tbl:error_estimates_supg}
    
    \subcaption{First order elements.}
    \begin{tabular}{ccccc}
        \toprule
        {$\mu$} &    $\alpha$ &    $C_\alpha$ &      $\beta$ &    $C_\beta$ \\
        \midrule
        1.00 &  1.024002 &  0.068333 &  0.988169 &  0.348477 \\
        0.10 &  0.928208 &  0.817010 &  0.808787 &  5.626513 \\
        0.01 &  0.714465 &  0.919312 &  0.197124 &  9.027664 \\
        \bottomrule
    \end{tabular}\vspace{2em}
    \subcaption{Second order elements.}
    \begin{tabular}{ccccc}
        \toprule
        {$\mu$} &    $\alpha$ &    $C_\alpha$ &      $\beta$ &    $C_\beta$ \\
        \midrule
        1.00 &  -5.295546 &  0.000071 &  -6.289647 &  0.000438 \\
        0.10 &  -7.960629 &  0.000000 &  -9.286106 &  0.000000 \\
        0.01 &   0.609334 &  0.719865 &  -0.245373 &  3.387804 \\
        \bottomrule
    \end{tabular}
\end{table}
\end{document}
